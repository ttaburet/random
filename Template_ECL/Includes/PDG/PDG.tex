% !TEX root =  ../main.tex
\makeatletter
\def\@ecole{école}
\newcommand{\ecole}[1]{
  \def\@ecole{#1}
}

\def\@specialiteFR{Spécialité}
\newcommand{\specialiteFR}[1]{
  \def\@specialiteFR{#1}
}

\def\@ED{\'{E}cole Doctorale}
\newcommand{\ED}[1]{
  \def\@ED{#1}
}

\def\@doctorat{Doctorat}
\newcommand{\doctorat}[1]{
  \def\@doctorat{#1}
}

\def\@adresse{Adresse}
\newcommand{\adresse}[1]{
  \def\@adresse{#1}
}

\def\@titleFR{}
\newcommand{\titleFR}[1]{
  \def\@titleFR{#1}
}
\def\@dateFR{}
\newcommand{\dateFR}[1]{
  \def\@dateFR{#1}
}
\def\@building{}
\newcommand{\building}[1]{
  \def\@building{#1}
}
\def\@room{}
\newcommand{\room}[1]{
  \def\@room{#1}
}
\def\@hour{}
\newcommand{\hour}[1]{
  \def\@hour{#1}
}
\def\@resumeFR{}
\newcommand{\resumeFR}[1]{
  \def\@resumeFR{#1}
}
\def\@resumeLongFR{}
\newcommand{\resumeLongFR}[1]{
  \newpage
    \chapter*{Résumé en français}\label{chap:resumeLongFR}
      \addcontentsline{toc}{chapter}{Résumé en français} \markboth{Résumé en français}{}
        {#1}
}
\def\@titleENG{}
\newcommand{\titleENG}[1]{
  \def\@titleENG{#1}
}
\def\@dateENG{}
\newcommand{\dateENG}[1]{
  \def\@dateENG{#1}
}
\def\@resumeENG{}
\newcommand{\resumeENG}[1]{
  \def\@resumeENG{#1}
}
\def\@resumeLongENG{}
\newcommand{\resumeLongENG}[1]{
    \newpage
    \chapter*{Résumé en anglais}\label{chap:resumeLongENG}
      \addcontentsline{toc}{chapter}{Résumé en français} \markboth{Résumé en anglais}{}
        {#1}
}
\def\@keywordsFR{}
\newcommand{\keywordsFR}[1]{
  \def\@keywordsFR{#1}
}

\def\@keywordsENG{}
\newcommand{\keywordsENG}[1]{
  \def\@keywordsENG{#1}
}

\def\@laboratory{}
\newcommand{\laboratory}[1]{
  \def\@laboratory{#1}
}

\def\@directeur{directeur}
\newcommand{\directeur}[1]{
  \def\@directeur{#1}
}

\def\@encadrant{encadrant}
\newcommand{\encadrant}[1]{
  \def\@encadrant{#1}
}

\def\@jury{}
\newcommand{\jury}[1]{
  \def\@jury{
    \begin{tabular}{>{\bfseries}lll}
      #1
    \end{tabular}
    \vfill
  }
}

\makeatother

\makeatletter
\newcommand{\pagedegarde}{
  \begin{figure}[t]
    \vspace*{-1in}
    \hspace*{1in}
    \includegraphics[width=0.35\textwidth, right]{Includes/PDG/logos/logo_Centrale_Lille.png}
  \end{figure}

  \vspace*{5em} %25em
  \colorbox{lightgray}{
    \begin{tabular*}{\textwidth}{l @{\extracolsep{\fill}} l}
      \textbf{Soutenance le} \@dateFR & \textbf{à} \@hour \\
      \textbf{Bâtiment} \@building & \textbf{Salle} \@room
    \end{tabular*}
  }
  % \begin{flushleft}
  %   %\textbf{N\textdegree d'ordre:} \@orderNumer\\
  % \end{flushleft}
  \begin{centering}
    \vspace*{1em}
    \textbf{\uppercase{Centrale Lille}}\\
    \vspace*{1em}
    \textbf{\uppercase{Thèse}}

    Présentée en vue \\
    d'obtenir le grade de

    \textbf{\uppercase{Docteur}}

    En

    \textbf{Spécialité:} \@specialiteFR
    \vspace*{1em}\\
    Par

    {\large\@author}

    \vspace*{1em}
    \textbf{\uppercase{Doctorat délivré par Centrale Lille}}
    \vspace*{1em}

    Titre de la thèse:\\
    \textbf{{\large\@titleFR}}\\
    \vspace{2em}

    Soutenue le \@dateFR \space devant le jury d'examen:\\
    \vspace{5mm}
    \@jury

    \vspace*{1em}
    Thèse préparée dans le Laboratoire \@laboratory\\

    \begin{figure}[H]
      \begin{centering}
      \subfloat{\begin{centering}
      \includegraphics[clip,width=0.2\textwidth]{Includes/PDG/logos/logo_Centrale_Lille.png}
      \par\end{centering}
      }\subfloat{\begin{centering}
      \includegraphics[width=0.2\textwidth]{Includes/PDG/logos/logo_Univ_Lille.png}
      \par\end{centering}
      }\subfloat{\begin{centering}
      \includegraphics[width=0.2\textwidth]{Includes/PDG/logos/logo_Cristal.png}
      \par\end{centering}
      }
      \par\end{centering}
    \end{figure}
  \end{centering}
	
  \cleardoublepage
  
  \newpage
	\begin{flushleft}
    \textbf{{\large Titre en français}}:\\ \@titleFR\\
    \vspace*{1em}
    \textbf{{\large Résumé en français}}:\\ \@resumeFR

    \vfill
    Mots-clefs: \@keywordsFR\\
  \end{flushleft}

  \newpage
  \begin{flushleft}
    \textbf{{\large Titre en anglais}}:\\ \@titleENG\\
    \vspace*{1em}
    \textbf{{\large Résumé en anglais}}:\\ \@resumeENG

    \vfill
    Mots-clefs: \@keywordsENG\\
  \end{flushleft}

\restoregeometry  
}